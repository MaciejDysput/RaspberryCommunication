\section{Zakończenie}
W ramach wykonania postawionych celów projektu zaprojektowano aplikację webową. Pełniła ona rolę czatu, która pozwalała na komunikację tekstową pomiędzy człowiekiem, a maszyną. W aplikacji zastosowano dodatkowe technologie frontendu i backendu co pozwoliło na wygodną obsługę programu przez użytkownika. Poprzez wysyłanie odpowiednich komend, możliwe było przeprowadzenie komunikacji z urządzeniem jakim było Raspberry Pi. W aplikacji zaimplementowano metody, które pozwalały na sterowanie elementami wykonawczymi podłączonymi do odpowiednich pinów GPIO mikrokontrolera. Poprzez skonfigurowanie protokołu SSH i stworzenie statycznego adresu IP możliwe było zdalne kontrolowanie urządzenia. Istotną rolę odegrała również implementacja komunikacji klienta z serwerem. Wykonanie tej operacji umożliwiło wymianę danych z mikrokontrolerem w czasie rzeczywistym. Pozwoliło to na wysyłanie poleceń i uzyskanie odpowiedzi bez żadnych zwłok czasowych. Przeprowadzone testy utwierdziły poprawność zaimplementowanego komunikatora. Wszystkie założenia zawarte w projekcie zostały spełnione i program można uznać za stabilny. Aplikację również można rozbudować o dalsze funkcjonowania. Zarówno stronę klienta jak i serwera można rozwinąć o różne czynności. Sam interfejs użytkownika może zostać rozbudowany o technologie rozwijające funkcjonalność aplikacji, a także ułatwiające obsługę przez użytkownika. Również można rozbudować strukturę układu połączonego z mikrokontrolerem. Poprzez zaimplementowanie odpowiednich funkcji po stronie serwera, możliwe będzie zastosowanie więcej operacji podczas komunikacji tekstowej. W tych aspektach, aplikacje webowe są nieograniczone i mogą zostać rozwinięte do poziomu nieprzekraczającego granic wyobraźni człowieka.