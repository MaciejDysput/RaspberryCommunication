Celem rozdziału jest przedstawienie zależności komunikacji klient-serwer oraz ukazanie ich przeznaczenia do tworzenia aplikacji webowych. 
\section{Socket.IO}
Socket.IO jest to biblioteka Javascript, która pozwala na realizację połączeń z serwerem przy użyciu WebSocket. Ten protokół sieciowy umożliwia przesyłanie danych w czasie rzeczywistym. Dane mogą być przesyłane z strony kliena na stronę serwera oraz na odwrót. Przy zastosowaniu biblioteki Socket.IO jesteśmy w stanie tworzyć aplikacje, które potrzebują wymianę danych w czasie rzeczywistym, a także wymagają nieprzerwanej komunikacji pomiędzy obiema stronami. Obie te cechy idealnie pasują do stworzenia komunikatora, w którym przesyłanie danych powinno odbywać się w czasie rzeczywistym, a także umożliwi nam nieprzerwaną komunikację z drugą stroną. Socket.IO może korzystać z metody emitowania komunikatów, które pozwolą na wysyłanie i odbieranie danych, a także nasłuchiwania takie jak połączenie bądź rozłączenie klienta. Jeżeli chcemy by nasza aplikacja umożliwiała przesyłanie danych w czasie rzeczywistym musimy korzystać z Socket, ponieważ żądania HTTP są bardzo powolne przez co uniemożliwiają nam komunikację w czasie rzeczywistym.
\section{Klient}
W komunikacji klient-serwer z użyciem Socket.IO, klient odpowiedzialny jest za nawiązanie połączenia z serwerem i wysyłanie oraz odbieranie danych za pomocą Socket.IO. W aplikacji może być to realizowane za pomocą skryptów Javascript, które są uruchamiane po stronie klienta. Klient może nawiązać połączenie z serwerem, ustawić odpowiednie obsługi zdarzeń czy również odbierać dane z serwera i je wysyłać. Oprócz nawiązywania połączenia i przesyłania danych, klient jest również odpowiedzialny za obsługę błędów połączenia oraz zamknięcia połączenia. Podczas komunikacji każdy klient posiada swój identyfikator gniazda.
\section{Serwer}
W komunikacji klient-serwer z użyciem Socket.IO, serwer odpowiedzialny jest za utrzymanie połączenia sieciowego z klientem i przesyłanie oraz odbieranie danych za pomocą Socket.IO. W aplikacji może być to realizowane za pomocą skryptów Javascript, które są uruchamiane na serwerze. Serwer może utworzyć instancję Socket.IO i ustawić ją na odpowiednim adresie URL, a następnie ustawić obsługę zdarzeń dla połączeń klientów. Serwer również może odbierać dane od klienta, a także wysyłać dane do klienta. Oprócz utrzymywania połączenia i przesyłania danych, serwer jest również odpowiedzialny za obsługę błędów połączenia oraz zamknięcia połączenia. Serwer może odpowiadać na żądania i przetwarzać dane przesyłane przez klienta. Każdy serwer posiada listę podłączonych gniazd, które korzystają z jego usług. Za pomocą tej listy jest w stanie wybrać identyfikator klienta, który się z nim komunikuje i przesłać mu dane.
\section{Komunikacja klient-server}
Komunikacja klient-serwer z użyciem Socket.IO jest stosowana w aplikacji, które wymagają szybkiej i bezstanowej wymiany danych pomiędzy klientem, a serwerem. Może to być przydatne w przypadku aplikacji, w których dane muszą być wymieniane w czasie rzeczywistym, aby zapewnić płynne działanie aplikacji. Socket.IO umożliwia tworzenie połączeń sieciowych co pozwala na elastycznie dostosowywanie sposobu komunikacji do potrzeb aplikacji. Aplikacja webowa do komunikacji tekstowej człowiek-maszyna korzysta z biblioteki Socket.IO. Umożliwia ona tworzenie aplikacji klient-serwer za pomocą połączeń sieciowych takich jak WebSockets. Biblioteka Socket.IO może być używana zarówno po stronie klienta jak i na stronie serwera. Model komunikacji klient-serwer umożliwia przesyłanie danych na różne komputery połączone w sieci. Czynność ta jest możliwa w momencie kiedy istnieją procesy wysyłające żądania czyli klienci, a także procesy, które je obsługują czyli serwery. Przy takim podziale obowiązków, aplikacja umożliwia użytkownikom dostęp do przesłanych i udostępnionych danych. Główną zaletą takiej komunikacji jest przede wszystkim obsługa dowolnej liczby klientów niezależnie od ich lokalizacji. Serwer może zostać postawiony na dowolnej maszynie w sieci i stąd będzie udostępniał wszystkie dane klientom. W aplikacji została zastosowana architektura dwuwarstwowa. Umożliwia ona korzystanie z serwera dowolnej liczbie klientów. Dwukierunkową komunikację w czasie rzeczywistym umożliwia biblioteka Socket.IO, która asynchronicznie aktualizuje przesyłane przez nas żądania do serwera. 
\begin{figure}
	\centering
	\includegraphics[width=0.7\linewidth]{"obrazy/Żądanie"}
	\caption{Żądanie i odpowiedź komunikacji Socket.IO}
	\label{fig:5}
\end{figure}
