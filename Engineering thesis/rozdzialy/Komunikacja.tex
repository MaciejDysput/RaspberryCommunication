Celem rozdziału jest przedstawienie sposobu komunikacji pomiędzy poszczególnymi częściami aplikacji. Ponadto zostały w nim omówione biblioteki, które umożliwiły implementację wymiany danych oraz protokół przesyłania informacji.
\section{Socket.IO}
Socket.IO to biblioteka Javascript, która pozwala na realizację połączeń z serwerem przy użyciu WebSocket. Można ją użyć zarówno do odbierania wiadomości jako odbiorca i ich wysyłania jako nadawca. Zastosowaniem biblioteki jest przesyłanie danych w czasie rzeczywistym. Zapewni to nieprzerwaną komunikację pomiędzy obiema stronami. Obie te cechy idealnie pasują do stworzenia komunikatora, w którym przesyłanie danych powinno odbywać się w czasie rzeczywistym, a także umożliwi nam nieprzerwaną komunikację z drugą stroną. Socket.IO może korzystać z metody emitowania komunikatów, które pozwolą na wysyłanie i odbieranie danych, a także nasłuchiwania takie jak połączenie bądź rozłączenie klienta. Alternatywnym rozwiązaniem dla Socket.IO jest wykorzystanie żądań HTTP do przesyłania danych do serwera, a następnie dystrybucja ich do podłączonych użytkowników. Jednakże, rozwiązanie te gorzej sprawdza się ze względu na złożoność protokołu HTTP jak i problemu do rozwiązania, w porównaniu do dwu-kierunkowej nieprzerwanej komunikacji pomiędzy klientami. Dodatkowo, komunikaty wysyłane przez WebSocket są znacznie mniejsze. W związku z tym w znaczący sposób wpłynie to na szybkość komunikacji w czasie rzeczywistym między użytkownikami oraz maszyną.

\section{Klient}W komunikacji klient-serwer z użyciem Socket.IO, klient odpowiedzialny jest za nawiązanie połączenia z serwerem oraz jego zarządzanie. Dodatkowo, za jego pomocą, użytkownik może wysłać oraz odbierać dane nadane przez innych nadawców. W aplikacji może być to realizowane za pomocą skryptów Javascript, które są uruchamiane po stronie klienta. Klient może nawiązać połączenie z serwerem, ustawić odpowiednie obsługi zdarzeń czy również odbierać dane z serwera i je wysyłać. Stworzone połączenie websocket pozwala na przechwycenie informacji o potencjalnych błędach w komunikacji. W tym przypadku, aplikacja po napotkaniu błędu, może powtórzyć wysłanie wiadomości lub wyświetlić komunikat błędu oraz poprawnie obsłużyć wyjątek. Inicjalizacja połączenia generuje unikalny identyfikator gniazda.

\section{Serwer}
W komunikacji klient-serwer z użyciem Socket.IO, serwer odpowiedzialny jest za utrzymanie połączenia sieciowego z klientem oraz jego zarządzanie. Aplikacja uruchomiona na mikrokontrolerze, posiada skrypty umożliwiające połączenie się klientów oraz inicjowanie sesji komunikacji między nimi. Zostało to zrealizowane dzięki stworzeniu serwera HTTP przy użyciu biblioteki Express.JS oraz jego integracji z użyciem Socket.IO. Dzięki czemu, użytkownik wysyłający żądanie pod odpowiedni adres serwera, może po ustanowieniu połączenia komunikować się z innymi. Serwer również może odbierać dane od klienta, a także wysyłać dane do klienta. Oprócz utrzymywania połączenia i przesyłania danych, jest on również odpowiedzialny za obsługę jego zamknięcia - niezależnie czy przez napotkany wyjątek, czy na żądanie użytkownika. Serwer może odpowiadać na żądania i przetwarzać dane przesyłane przez klienta. Każdy serwer posiada listę podłączonych gniazd, które korzystają z jego usług. Za pomocą tej listy jest w stanie wybrać identyfikator klienta, który się z nim komunikuje i przesłać mu dane.
\section{Komunikacja klient-server}
Komunikacja klient-serwer z użyciem Socket.IO jest stosowana w aplikacji, które wymagają szybkiej i bezstanowej wymiany danych pomiędzy klientem, a serwerem. Może to być przydatne w przypadku aplikacji, w których dane muszą być wymieniane w czasie rzeczywistym, aby zapewnić płynne działanie aplikacji. Socket.IO umożliwia tworzenie połączeń sieciowych co pozwala na elastycznie dostosowywanie sposobu komunikacji do potrzeb aplikacji. Model komunikacji klient-serwer umożliwia przesyłanie danych na różne komputery połączone w sieci. Czynność ta jest możliwa w momencie kiedy istnieją procesy wysyłające żądania czyli klienci, a także procesy, które je obsługują czyli serwery. Przy takim podziale obowiązków, aplikacja umożliwia użytkownikom dostęp do przesłanych i udostępnionych danych. Główną zaletą takiej komunikacji jest przede wszystkim obsługa dowolnej liczby klientów niezależnie od ich lokalizacji. Serwer może zostać skonfigurowany na dowolnej maszynie w sieci i stąd będzie udostępniał wszystkie dane klientom. W aplikacji została zastosowana architektura dwuwarstwowa. Umożliwia ona korzystanie z serwera dowolnej liczbie klientów. Dwukierunkową komunikację w czasie rzeczywistym umożliwia biblioteka Socket.IO, która asynchronicznie aktualizuje przesyłane przez nas żądania do serwera. 

