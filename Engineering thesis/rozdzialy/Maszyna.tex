Celem rozdziału jest przedstawienie zastosowanego sprzętu, zintegrowane z aplikacją w celach komunikacji tekstowej.
\section{Raspberry Pi 2 Model B} 
\begin{figure}[htbp]
	\centering
	\includegraphics[width=0.7\linewidth]{"obrazy/Raspberry Pi"}
	\caption{Raspberry Pi 2 Model B.}
	\label{fig:1}
\end{figure}
W aplikacji jako maszynę przyjęto mikrokomputer Raspberry Pi 2 Model B. To druga generacja Raspberry Pi, która opiera się na układzie system-on-chip BCM2836. Raspberry Pi 2 może być używany do przetwarzania tekstu czy programowania. W porównaniu do starszej wersji, posiada on procesor ARMv7. Oznacza to możliwość obsługi pełnej gamy dystrybucji ARM GNU/Linux,  w tym Snappy Ubuntu Core, a także Windows 10. Mikrokontroler wykorzystuje płytkę drukowaną rozmiaru 3,37 x 2,13 x 0,67 cala ważącą tylko 35 gramów. Są to mniej więcej rozmiary karty kredytowej. Raspberry Pi ogranicza obsługę pod względem systemu operacyjnego. Dostępny jest tylko Linux. Obecnie powszechnym i stosowanym systemem operacyjnym jest Windows, dlatego Raspberry Pi zmusza nas do poznania tajników wiersza poleceń Linuxa. W tym celu Fundacja Raspberry Pi stworzyła menadżer instalacji o nazwie NOOBS – New Out Of the Box Software. Instalator ten zawiera również różne systemy operacyjne innych firm, w tym OpenELEC oraz Windows 10 IoT Core firmy Microsoft. Raspberry Pi 2 nie ma wbudowanej pamięci co wiąże się z ograniczeniami do karty microSD i dowolnej dołączonej pamięci. Urządzenie posiada jeden czytnik karty microSD co powoduje możliwość korzystania z tylko jednego systemu operacyjnego.

\begin{table}[htbp]
\centering
\caption{Cechy urządzenia Raspberry Pi}
\begin{tabular}{|c|}
\hline
Czterordzeniowy procesor ARM Cortex-A7 900 MHz \\
\hline
1GB RAM \\
\hline
Rdzeń graficzny VideoCore IV 3D \\
\hline
Port Ethernet \\
\hline
Cztery porty USB i możliwość podłączenia klawiatury i myszki \\
\hline
Pełnowymiarowe wyjście HDMI i możliwość podłączenia się do monitora \\
\hline
Czterobiegunowe gniazdo 3,5mm z wyjściem audio i kompozytowym wyjściem wideo \\
\hline
40 stykowe złącze GPIO z męskimi stykami, które są kompatybilne z żeńskimi złączami \\
\hline
Interfejs kamery (CSI) \\
\hline
Interfejs wyświetlacza (DS1) \\
\hline
Gniazdo karty micro SD \\
\hline
\end{tabular}
\end{table}

\section{Złącze GPIO}

Raspberry Pi jest przystosowane do pracy z napięciem 3,3V oraz z maksymalnym obciążeniem prądowym równym 16mA. Podanie wyższego napięcia może wiązać się z uszkodzeniem mikrokomputera. Płytka zawiera 40-pinowe złącze. Piny numer 2 i 4 podłączone są do 5V. Jeżeli dokonamy zwarcia tych pinów z innymi pinami możemy doprowadzić do uszkodzenia płytki Raspberry Pi. Piny 6,9,14,20,25,30,34 oraz 39 podłączone są do masy. Raspberry Pi posiada dwukierunkowe piny wejście/wyjście. Sterowanie liniami wejście/wyjście wymaga zaprogramowania w wybranym przez nas języku. Istnieją dwa różne schematy numeracji pinów. Piny GPIO są cyfrowymi pinami co znaczy, że posiadają stan on – włączenia oraz off – wyłączenia. Za pomocą pinów masy – GND, jesteśmy w stanie zmierzyć wszystkie napięcia i zakończyć obwód elektryczny. Pin GND traktuje się jako nasz punkt zerowy i podczas projektowania układów ważną jest rzeczą skorzystanie z pinu uziemiającego przed podaniem napięcia na obwód. Dzięki takiej operacji unikniemy uszkodzenie najbardziej wrażliwych elementów w układzie. Przy podłączeniu dowolnego komponentu ze źródłem zasilania i uziemieniem, element staje się częścią naszego obwodu dzięki czemu poprzez przepływ jesteśmy uzyskać wymagany przez nas efekt. W przypadku podłączenia diody LED wytworzymy światło. Piny 3V zapewniają zasilanie naszego układu napięciem 3,3V. Przy podłączeniu diody LED do GPIO 3V możemy sprawdzić czy nasz układ został prawidłowo podłączony. Należy również brać pod uwagę fakt, że pin 3V ma ograniczone możliwości prądowe. Wartość jaką można pobrać to około 50mA. Oznacza to, że nie jesteśmy w stanie obsłużyć naszą płytkę dużymi prądami. Jeżeli chcemy podłączyć elementy, które wymagają większej wartości prądu, należy pomyśleć nad innym sposobem zasilania. Piny 3V zawierają również ochronę przeciwzwarciową. Jeżeli zostanie podłączony element, który pobiera większy prąd niż jest wymagany to zabezpieczenie automatycznie zareaguje i odciąży pin. Tym samym unikniemy uszkodzenia naszej płytki Raspberry Pi. Pin GPIO 3V jest zasilany napięciem stałym, a nie zmiennym, dlatego nie nadaje się do sterowania takich urządzeń jak silniki czy przekaźniki. Piny 3V najlepiej sprawdzają się przy zasilaniu takich elementów jak diody LED czy czujniki, ponieważ elementy te charakteryzują się niskim poborem mocy. Piny 5V są typem pinów, które pozwalają na zasilanie urządzeń zewnętrznych za pośrednictwem płytki Raspberry Pi. Są one zasilane napięciem stałym o wartości nieprzekraczającej 5V i może dostarczać maksymalnie do 2,5A prądu. Pozwalają one zasilać urządzenia o wyższym poborze mocy. Również w złączu GPIO, znajdują się piny korzystające z protokołów I2C, SPI lub UART. Magistrala I2C to cyfrowy interfejs komunikacyjny. Umożliwia przesyłanie danych pomiędzy urządzeniami za pomocą linii SDA (Serial Data) oraz SCL (Serial Clock). Piny o numerze 3 i 5 należą do protokołu I2C. Pin zegarowy SCL umożliwia synchronizację przesyłania danych pomiędzy urządzeniami. Linia SDA pozwala na przesyłanie danych pomiędzy urządzeniami za pomocą magistrali I2C. Głównym zastosowaniem tych pinów jest możliwość podłączenia takich urządzeń jak ekrany LCD. Magistrala I2C wymaga obecności przynajmniej dwóch urządzeń, które jedno będzie pełniło funkcję master, a drugie slava. Urządzenie Master będzie wysyłało komendy do slave’a. Raspberry Pi jest wyposażone w 4 piny SPI, które umozliwiają komunikację między urządzeniami. Są to MOSI, MISO, SCLK oraz CE0 lub CE1. Pin MOSI umożliwia nam przesyłanie danych od mastera do slave’a, zaś MISO przesyła dane z slave’a do mastera. Zegarowy pin SCLK umożliwia synchronizację przesyłania danych pomiędzy urządzeniami. Pin CE0 lub CE1 służy do włączania lub wyłączania slave’a. Raspberry Pi wyposażone jest w dwa piny UART. Pierwszy to RX (Receive), a drugi to TX (Transmit). Pin RX służy do odbierania danych z urządzenia zewnętrznego, zaś pin TX pozwala na wysyłanie danych do urządzenia zewnętrznego. 
\begin{figure}[htbp]
	\centering
	\includegraphics[width=0.5\linewidth]{"obrazy/GPIO"}
	\caption{40stykowe złącze GPIO Raspberry Pi 2 Model B.}
	\label{fig:2}
\end{figure}
\newpage
\section{Płytka Prototypowa}

Płytka prototypowa to narzędzie umożliwiające tworzenie projektów elektronicznych. Pozwala na łatwe połączenie elementów wykonawczych takich jak diody, rezystory czy serwomechanizmy. W skład konstrukcji płytki wchodzą takie elementy jak:
\begin{itemize}  
	\item \textbf{Szyna zasilania}, używana jest do dostarczenia napięcia zasilającego do płytki prototypowej. Jest to niezbędne do działania wszystkich elementów podłączonych na płytce.
Może obejmować zarówno napięcia stałe jak i zmienne. Linia ta zazwyczaj oznaczona jest literami „VCC” lub „+”. 
	\item \textbf{Szyna masy}, używana jest jako punkt odniesienia dla innych sygnałów elektrycznych w układzie. Wszystkie inne napięcia na płytce są względem niej mierzone. Linia ta zwykle oznaczona jest symbolem „GND” lub „-„. Pozwala na uzyskanie stabilnego poziomu napięcia referencyjnego. Jest ono niezbędne do prawidłowego działania elementów elektronicznych umieszczonych na płytce. Bez podłączenia szyny masy, obiekty mogłyby działać na różnych poziomach napięcia. Mogłoby to doprowadzić do błędów w działania układu.
	\\
\item \textbf{Punkty wiązania}, służą do połączenia elementów elektronicznych na płytce prototypowej. Odpowiadają za przesyłanie sygnałów elektrycznych między obiektami. Umożliwia to pracę całego układu. Za pomocą punktów wiązania możliwe jest dokonywanie połączeń elementów elektronicznych z szyną zasilającą i szyną masy.


\end{itemize}



