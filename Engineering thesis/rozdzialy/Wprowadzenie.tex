
Początki komunikacji tekstowej pomiędzy człowiekiem, a maszyną sięgają kilkudziesięciu lat wstecz. Ich długa historia sięga od czasów początków komputera kiedy to maszyny te były obsługiwane przez operatorów. Komunikowali się oni z komputerem poprzez wpisywanie różnych danych za pomocą klawiatury. W późniejszych latach XX wieku pojawiły się różnego rodzaju systemy komunikacji tekstowej jak poczta elektroniczna, która zastępowała tradycyjną pocztę, a także komunikatory tekstowe, które ułatwiały komunikację pomiędzy ludźmi. Również komunikacja tekstowa pomiędzy człowiekiem, a maszyną znalazła swoje zastosowanie w różnych działach automatyki. Obejmowało to różnego rodzaju komunikacji, takie jak polecenia, pytania, odpowiedzi i komunikaty błędów. W automatyce, komunikację tekstową zaczęto stosować przede wszystkim by umożliwić komunikację ludziom z maszynami, które brały udział w różnych procesach przemysłowych czy wytwórczych. Do przesłanej wiadomości bądź żądania do maszyny, człowiek uzyskiwał określoną informację zwrotną. Komunikacja ta najczęściej stosowana była za pomocą interfejsów tekstowych lub graficznych. W automatyce przemysłowej komunikacja tekstowa jest ważnym narzędziem do zarządzania i sterowania procesami, monitorowania stanu maszyn a także samej kalibracji urządzeń i dokonywania wymaganych prac diagnostycznych. Informacje te przeważnie przekazywane są w postaci języka naturalnego. Za pomocą komunikacji tekstowej, maszyna jest w stanie rozpoznać i zrozumieć wprowadzony przez nas tekst i zamienić go na konkretne działania. Komunikacja tekstowa pomiędzy człowiekiem, a maszyną do dnia dzisiejszego ma szerokie zastosowanie. Umożliwia ona ludziom przesyłanie danych bądź korzystanie z różnych aplikacji i usług online takie jak czaty czy forum internetowe. Komunikacja tekstowa jest dostępna w każdym miejscu i o każdej porze tylko wymagany jest dostęp do urządzenia. Główną z zalet takiej komunikacji jest brak barier językowych. Komunikacja tekstowa umożliwia komunikację z maszynami, które są programowane do obsługi wielu języków przez co przekazywanie informacji może odbywać się z osobami posługującymi się innymi językami.
\newpage
\section{Cel pracy}
Celem pracy jest stworzenie aplikacji internetowej, która umożliwi użytkownikom komunikację zdalną z mikrokontrolerem. Poprzez zaimplementowanie odpowiednich funkcji, możliwe będzie uzyskanie odpowiedzi od maszyny za pomocą wpisanych komend w komunikatorze. Aby uzyskać i monitorować otrzymane wyniki, zbudowano układ działania, w którego skład wchodzą elementy wykonawcze takie jak diody LED, rezystory i serwomechanizm.

\section{Zakres pracy}
Do realizacji założeń pracy i stworzenia aplikacji internetowej wykonano: 
\begin{itemize}  
	\item translację komend wpisanych w języku naturalnym na odpowiednie funkcje na mikrokontrolerze w aplikacji będącej komunikatorem
	\\
	\item wsparcie dla komend uruchamiających elementy wykonawcze, np. dioda LED, wiatrak - sekwencyjne wykonywanie komend
	\\
	\item sekwencyjne wykonywanie komend
	\\
	\item archiwum poleceń wpisanych do aplikacji
	\\
	\item wykorzystanie dowolnych technologii frontendu i backendu typowych dla aplikacji webowych
	\\
\end{itemize}