Celem rozdziału jest przedstawienie możliwości konfiguracji Raspberry Pi z komputerem zapewniając zdalną komunikację między urzązeniami.
\section{Statyczny adres IP}
Adres IP przypisuje określony adres do urządzenia, który się z nim łączy. Jego tendencją jest to, że się zmienia z biegiem czasu w zależności jakie urządzenie się podłączyło. Adres IP jest potrzebny do identyfikacji urządzenia, które się podłączyło do sieci, aby ułatwić mu dostęp do korzystania z usługi. Otrzymywany jest unikalny adres IP w sieci. Naszym celem jest ustawienie takiego adresu IP urządzenia by te przy podłączeniu z routerem nie ulegało zmianie i się nie rozłączało. Do ustawienia trwałego połączenia sieciowego między Raspberry Pi, a komputerem potrzebne będzie ustawienie statycznego adresu IP. Używany jest, aby umożliwić stale dostęp do urządzenia z sieci lokalnej lub internetu. Ustawianie statycznego adresu IP może być szczególnie przydatne kiedy chcemy zdalnie obsługiwać Raspberry Pi za pomocą protokołu SSH oraz udostępniać dane między tymi urządzeniami. 
\section{Protokół SSH}
Jest to protokół sieciowy, który umożliwia bezpieczne połączenie się z komputerem i wykonywanie na nim zdalnie poleceń. Jest to niezbędny element przy konfiguracji systemu operacyjnego na Raspberry Pi. Umożliwia zdalne zarządzanie urządzeniem za pomocą komputera, w którym zainstalowany jest inny system operacyjny. Dzięki temu, jesteśmy w stanie dokonywać regularnej aktualizacji. Potrzebujemy do tego nawiązywania połączenia z urządzeniem przez sieć co protokół SSH nam zapewnia. W tej samej sieci korzystając z komputera jesteśmy w stanie uzyskać dostęp do wiersza poleceń Raspberry Pi. Protokół SSH jest bardzo przydatny w przypadku konfiguracji i zarządzania urządzeniami zdalnie, ponieważ umożliwia bezpieczne połączenie się z urządzeniem i wykonywanie poleceń na nim, unikając konieczności fizycznego dostępu do urządzenia. Protokół SSH w celach bezpieczeństwa szyfruje dane przesyłane między komputerami.