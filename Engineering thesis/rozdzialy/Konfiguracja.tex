Celem rozdziału jest przedstawienie możliwości konfiguracji Raspberry Pi z komputerem potrzebnej do zdalnej komunikacji między urządzeniami.
\section{Statyczny adres IP}
Adres IP jest adresem, który przypisuje do dowolnego urządzenia komputer, który się z nim łączy. Tendencją tego adresu IP jest to, że się zmienia między urządzeniami z biegiem czasu w zależności jakie urządzenie się podłączyło. Adres IP jest potrzebny do identyfikacji urządzenia, które się podłączyło do sieci, aby ułatwić mu dostęp do korzystania z usługi. Podczas podłączenia, urządzenie otrzymuje unikalny adres IP w sieci. Naszym celem jest ustawienie takiego adresu IP urządzenia by te przy podłączeniu z routerem nie ulegało zmianie i się nie rozłączało. Do ustawienia trwałego połączenia sieciowego między Raspberry Pi, a komputerem potrzebne będzie ustawienie statycznego adresu IP. Statyczny adres IP jest używany, aby umożliwić stale dostęp do urządzenia z sieci lokalnej lub Internetu. Może być używany w przypadku urządzenia takiego jak Raspberry Pi, które może być uznawane jako serwer lub do innych celów wymagających stałego dostępu do sieci. Ustawianie statycznego adresu IP może być szczególnie przydatne kiedy chcemy zdalnie obsługiwać Raspberry Pi za pomocą protokołu SSH oraz udostępniać dane między tymi urządzeniami. 
\section{Protokół SSH}
Jest to protokół sieciowy, który umożliwia bezpieczne połączenie się z komputerem i wykonywanie na nim zdalnie poleceń. Jest to niezbędny element przy konfiguracji systemu operacyjnego na Raspberry Pi, ponieważ umożliwia on zdalne zarządzanie urządzeniem za pomocą komputera, w którym zainstalowany jest inny system operacyjny. Za pomocą zdalnego dostępu do urządzenia jesteśmy w stanie dokonywać regularnej aktualizacji. Potrzebujemy do tego nawiązywania połączenia z urządzeniem przez sieć co protokół SSH nam zapewnia. W tej samej sieci korzystając z komputera jesteśmy w stanie uzyskać dostęp do wiersza poleceń Raspberry Pi. Zagwarantuje nam to zdalne zarządzanie mikrokomputerem. Protokół SSH jest bardzo przydatny w przypadku konfiguracji i zarządzania urządzeniami zdalnie, ponieważ umożliwia bezpieczne połączenie się z urządzeniem i wykonywanie poleceń na nim, unikając konieczności fizycznego dostępu do urządzenia. Jest również często używany do zabezpieczania połączeń sieciowych, ponieważ protokół SSH szyfruje dane przesyłane między komputerami.